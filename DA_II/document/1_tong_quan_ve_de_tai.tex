\chapter{Tổng quan về đề tài}
  \section{Lý do chọn đề tài}
  Mỗi năm Đại học Bách khoa Hà Nội tuyển sinh khoảng gần 8000 sinh viên (số liệu năm 2022), với nhu cầu phục vụ các sinh viên một cách đầy đủ nhất. 
  Công tác quản lý sinh viên (kết quả học tập) của sinh viên đóng vai trò hết sức quan trọng đối với hoạt động của một khoa/viện trong các trường đại học. \\
  
  Đánh giá kết quả học tập của Đại học Bách khoa Hà Nội:
  \begin{itemize}
    \item[1.] Kết quả học tập trong một học kỳ của sinh viên được đánh giá trên cơ sở điểm của các học phần thuộc chương trình đào tạo không kể các học phần có điểm R và các môn yêu cầu chứng chỉ riêng (Ngoại ngữ, Giáo dục thể chất, Giáo dục quốc phòng-an ninh), thể hiện bằng các chỉ số:
      \begin{itemize}
        \item[a.] Số tín chỉ đạt là tổng số tín chỉ của các học phần có điểm đạt trong học kỳ.
        \item[b.] Số tín chỉ không đạt là tổng số tín chỉ của các học phần có điểm không đạt trong học kỳ.
        \item[c.] Điểm trung bình học kỳ (GPA) là trung bình cộng điểm số quy đổi theo thang 4 của các học phần mà sinh viên đã học trong học kỳ với trọng số là số tín chỉ của học phần. Điểm trung bình học kỳ được làm tròn tới 2 chữ số thập phân.  
      \end{itemize}
    \item[2.] Kết quả tiến bộ học tập của sinh viên từ đầu khoá được đánh giá trên cơ sở điểm của các học phần thuộc chương trình đào tạo không kể các môn yêu cầu chứng chỉ riêng, thể hiện bằng các chỉ số:
      \begin{itemize}
        \item[a.] Số tín chỉ tích luỹ (TCTL) là tổng số tín chỉ của các học phần đã đạt từ đầu khoá kể cẩ các học phần được miễn, được chuyển điểm.
        \item[b.] Số tỉn chỉ nợ tồn đọng là tổng số tín chỉ của các học phần đã học nhưng chưa đạt từ đầu khoá.
        \item[c.] Điểm trung bình tích luỹ (CPA) là trung bình cộng điểm số quy đổi theo thang 4 của các học phần đã học từ đầu khoá với trọng số là số tín chỉ của học phần. Điểm trung bình tích luỹ được làm tròn tới 2 chữ số thập phân.
        \item[d.] Trình độ ngoại ngữ của sinh viên đạt được theo yêu cầu chương trình đào tạo, thể hiện qua kết quả thi nội bộ trong trường và các chứng chỉ ngoại ngữ được xét tương đương.  
      \end{itemize}
    \item[3.] Sinh viên được xếp hạng trình độ năm học căn cứ số tín chỉ tích luỹ như sau:
    \item[4.] Sinh viên được xếp loại học lực theo học kỳ căn cứ điểm trung bình học kỳ và xếp loại học lực từ đầu khoá căn cứ điểm trung bình tích luỹ như sau:
  \end{itemize}
  \begin{tabular}{|l|l|}
    \hline
    \textbf{Các tác nhân}         & Admin, sinh viên, giảng viên, giáo vụ, trưởng phòng \\
    \hline
    \textbf{Mô tả}                & Đăng nhập                                           \\
    \hline
    \textbf{Kích hoạt}            & Người dùng nhấn vào nút "Đăng nhập" trên thanh menu \\
    \hline
    \textbf{Luồng chính}          & \makecell[l]{1. Chọn chức năng đăng nhập            \\ 2. Hiển thị màn hình đăng nhập \\ 3. Nhập tên đăng nhập và mật khẩu \\ 4. Hiển thị kiểm tra thông tin \\ 5. Nếu thành công chuyển tới giao diện chính \\ 6. Kết thúc} \\
    \hline
    \textbf{Các luồng thay thế}   & \makecell[l]{Mật khẩu không hợp lệ:                 \\ 1. Thông báo ra màn hình mật khẩu sai \\ 2. Quay lại bước 2 của luồng chính \\ Quên mật khẩu: \\ 1. Hiển thị màn hình nhập email \\ 2. Nhập và chọn chức năng quên mật khẩu \\ 3. Kiểm tra hợp lệ hệ thống gửi mail xác nhận \\ 4. Hiển thị thông báo thành công \\ 5. Kết thúc} \\
    \hline
    \textbf{Tiền điều kiện}       & Tài khoản trước đó đã được đăng ký                  \\
    \hline
    \textbf{Hậu điều kiện}        & Người dùng đăng nhập thành công                     \\
    \hline
    \textbf{Các yêu cầu đặc biệt} &                                                     \\
    \hline
  \end{tabular}
  

  Quản lí các dữ liệu có thể biết được những sinh viên đang chậm tiến độ học tập, những sinh viên đang bị mức cảnh cáo học tập để từ đó đưa ra giải pháp, tư vấn học tập phù hợp đối với các sinh viên.
  Ngoài ra, hiện nay các vấn đề về đăng ký làm đồ án, đăng ký thực tập cho sinh viên đang được diễn ra một cách thủ công dưới hình thức điền form truyền thống.
  Điều đó dẫn tới tốn rất nhiều thời gian để giảng viên sắp xếp, phản hồi, ... Vì vậy, em thực hiện đề tài "Phân tích, thiết kế hệ thống quản lý học tập và đăng ký đồ án, thực tập của sinh viên"
  với mong muốn giảm bớt khối lượng công việc cho cả Viện cũng như của sinh viên. Ngoài ra, hệ thống cũng giúp sinh viên dễ dàng hơn trong quá trình
  tìm hiểu cũng như chọn đề tài đồ án, vị trí thực tập của mình.

  \section{Yêu cầu của bài toán}
  %sửa
  Hệ thống cho phép quản lý 10000 người dùng bao gồm sinh viên, giảng viên, cán bộ và người quản trị hệ thống.
  Mỗi sinh viên sẽ sử dụng tài khoản microsoft mail trường cấp của mình để đăng ký tài khoản người dùng, sau đó có thể liên kết với
  các tài khoản xã hội khác (google gmail, facebook). \\
  
  Hệ thống quản lý sinh viên bao gồm các chức năng:
  \begin{itemize}
    \item Đăng ký, đăng nhập, đăng xuất khỏi hệ thống
    \item Cập nhật thông tin cá nhân (điểm học tập)
    \item Tra cứu thông tin đề tài đồ án, giảng viên hướng dẫn
    \item Tra cứu thông tin công ty thực tập
    \item Đăng ký đồ án
    \item Tạo CV cá nhân dành cho sinh viên
    \item Đăng ký thực tập
    \item Thống kê kết quả học tập dành cho giảng viên
  \end{itemize}

  \section{Tác nhân của hệ thống}
  \begin{tabular}{|c|c|l|}
    \hline
    \thead{STT} & \thead{Tác nhân} & \thead{Chức năng}                                                    \\
    \hline
    1           & Quản trị viên            & \makecell[l]{- Quản trị hệ thống.                                    \\ - Phân quyền người dùng. \\ - Cấp lại mật khẩu cho người dùng.}\\
    \hline
    2           & Cán bộ          & \makecell[l]{- Quản lý danh sách sinh viên.                          \\ - Quản lý form đăng ký trả về.} \\
    \hline
    3           & Giảng viên       & \makecell[l]{- Quản lý danh sách sinh viên lớp.}                     \\
    \hline
    4           & Sinh viên        & \makecell[l]{- Đăng nhập, đăng ký, đăng xuất, cập nhật thông tin.                        \\ - Sử dụng hệ thống, thực hiện điền form. \\ - Tìm kiếm, tra cứu thông tin.}\\
    \hline
    5           & Người dùng       & \makecell[l]{- Xem các thông tin về Viện, định hướng đào tạo, việc làm.} \\
    \hline
  \end{tabular}

  \section{Mô tả hệ thống}
    \subsection{Dành cho quản trị viên}
      \begin{itemize}
        \item Đăng nhập, đăng xuất khỏi hệ thống
        \item Phân quyền người dùng, quản trị hệ thống
      \end{itemize}
    \subsection{Dành cho sinh viên}
      \begin{itemize}
        \item 
      \end{itemize}
    \subsection{Dành cho giảng viên}
      \begin{itemize}
        \item 
      \end{itemize}
    \subsection{Dành cho cán bộ}
      \begin{itemize}
        \item 
      \end{itemize}