\chapter{Tổng quan về đề tài}
\section{Lý do chọn đề tài}
Mỗi năm Đại học Bách khoa Hà Nội tuyển sinh khoảng gần 8000 sinh viên (số liệu năm 2022), với nhu cầu phục vụ các sinh viên một cách đầy đủ nhất.
Công tác quản lý sinh viên (kết quả học tập) của sinh viên đóng vai trò hết sức quan trọng đối với hoạt động của một khoa/viện trong các trường đại học. \\

\textbf{Đánh giá kết quả học tập của Đại học Bách khoa Hà Nội:}\textsuperscript{\cite{daotao}}
\begin{itemize}
	\item[1.] Kết quả học tập trong một học kỳ của sinh viên được đánh giá trên cơ sở điểm của các học phần thuộc chương trình đào tạo không kể các học phần có điểm R (điểm học phần được miễn học và công nhận tín chỉ) và các môn yêu cầu chứng chỉ riêng (Ngoại ngữ, Giáo dục thể chất, Giáo dục quốc phòng-an ninh), thể hiện bằng các chỉ số:
		\begin{itemize}
			\item[a.] Số tín chỉ đạt là tổng số tín chỉ của các học phần có điểm đạt trong học kỳ.
			\item[b.] Số tín chỉ không đạt là tổng số tín chỉ của các học phần có điểm không đạt trong học kỳ.
			\item[c.] Điểm trung bình học kỳ (GPA) là trung bình cộng điểm số quy đổi theo thang 4 của các học phần mà sinh viên đã học trong học kỳ với trọng số là số tín chỉ của học phần. Điểm trung bình học kỳ được làm tròn tới 2 chữ số thập phân.
		\end{itemize}
	\item[2.] Kết quả tiến bộ học tập của sinh viên từ đầu khoá được đánh giá trên cơ sở điểm của các học phần thuộc chương trình đào tạo không kể các môn yêu cầu chứng chỉ riêng, thể hiện bằng các chỉ số:
		\begin{itemize}
			\item[a.] Số tín chỉ tích luỹ (TCTL) là tổng số tín chỉ của các học phần đã đạt từ đầu khoá kể cả các học phần được miễn, được chuyển điểm.
			\item[b.] Số tỉn chỉ nợ tồn đọng là tổng số tín chỉ của các học phần đã học nhưng chưa đạt từ đầu khoá.
			\item[c.] Điểm trung bình tích luỹ (CPA) là trung bình cộng điểm số quy đổi theo thang 4 của các học phần đã học từ đầu khoá với trọng số là số tín chỉ của học phần. Điểm trung bình tích luỹ được làm tròn tới 2 chữ số thập phân.
			\item[d.] Trình độ ngoại ngữ của sinh viên đạt được theo yêu cầu chương trình đào tạo, thể hiện qua kết quả thi nội bộ trong trường và các chứng chỉ ngoại ngữ được xét tương đương.
		\end{itemize}
	\item[3.] Sinh viên được xếp hạng trình độ năm học căn cứ số tín chỉ tích luỹ như sau:
    \begin{table}[h!]
      \begin{tabular}{|l|c|c|c|c|c|}
          \hline
          Số TCTL  & $<32$        & $32-63$     & $64-95$    & $96-127$   & $\geq 128$  \\
          \hline
          Trình độ & Năm thứ nhất & Năm thứ hai & Năm thứ ba & Năm thứ tư & Năm thứ năm \\
          \hline
        \end{tabular}   
    \end{table}	
	\item[4.] Sinh viên được xếp loại học lực theo học kỳ căn cứ điểm trung bình học kỳ và xếp loại học lực từ đầu khoá căn cứ điểm trung bình tích luỹ như sau:
    \begin{table}[h!]
      \begin{tabular}{|l|c|c|c|c|c|c|c|}
          \hline
          GPA/CPA  & $<1.0$ & $1.0-1.49$ & $1.5-1.99$ & $2.0-2.49$ & $2.5-3.19$ & $3.2-3.59$ & $3.6-4.0$ \\
          \hline
          Xếp loại & Kém    & Yếu        & TB yếu     & Trung bình & Khá        & Giỏi       & Xuất sắc  \\
          \hline
        \end{tabular}
    \end{table}	
\end{itemize}

\textbf{Xử lý kết quả học tập Đại học Bách Khoa Hà Nội:}\textsuperscript{\cite{daotao}} \\
Các hình thức xử lý kết quả học tập được áp dụng cuối mỗi học kỳ chính, bao gồm cảnh cáo học tập (mức 1 đến mức 3), buộc thôi học và hạn chế khối lượng đăng ký học tập.
\begin{itemize}
  \item[1.] Cảnh cáo học tập là hình thức cảnh cáo những sinh viên có kết quả học tập yếu kém, áp dụng cụ thể như sau:
    \begin{itemize}
      \item[a.] Nâng một mức cảnh cáo đối với sinh viên có số tín chỉ không đạt trong học kỳ lớn hơn 8.
      \item[b.] Nâng hai mức cảnh cáo đối với sinh viên có số tín chỉ không đạt trong học kỳ lớn hơn 16 hoặc tự ý bỏ học, không đăng ký học tập. 
      \item[c.] Áp dụng cảnh cáo mức 3 đối với sinh viên có số tín chỉ nợ tồn đọng từ đầu khoá lớn hơn 27.
      \item[d.] Sinh viên đang bị cảnh cáo học tập, nếu số tín chỉ không đạt trong học kỳ bằng hoặc nhỏ hơn 4 thì được hạ một mức cảnh cáo.  
    \end{itemize}
  \item[2.] Buộc thôi học là hình thức áp dụng đối với những sinh viên có kết quả quá trình học tập rất kém, cụ thể trong các trường hợp như sau:
    \begin{itemize}
      \item[a.] Sinh viên bị cảnh cáo học tập mức 3, ngoại trừ những đối tượng được hưởng chế độ ưu tiên trong đào tạo theo quy định của Bộ thì được nộp đơn xin gia hạn một học kỳ và chỉ một lần trong toàn khoá học.
      \item[b.] Sinh viên học vượt quá thời gian cho phép, hoặc không còn đủ khả năng tốt nghiệp trong thời gian cho phép theo quy định tại Điều 6.
    \end{itemize}
  \item[3.] Hạn chế khối lượng học tập là hình thức buộc những sinh viên học yếu kém hoặc chưa đạt chuẩn ngoại ngữ (xét tại thời điểm đăng ký học tập) đăng ký số tín chỉ học phần chuyên môn ít hơn bình thường, cụ thể như sau:
    \begin{itemize}
      \item[a.] Sinh viên bị cảnh cáo học tập mức 1 được đăng ký tối đa 18 TC và tối thiểu 10 TC cho một học kỳ chính.
      \item[b.] Sinh viên bị cảnh cáo học tập mức 2 được đăng ký tối đa 14 TC và tối thiểu 8 TC cho một học kỳ chính.
      \item[c.] Sinh viên không đạt chuẩn ngoại ngữ theo quy định cho từng trình độ năm học được đăng ký tối đa 14 TC và tối thiểu 8 TC cho một học kỳ chính.  
    \end{itemize}
\end{itemize}

Quản lí các dữ liệu có thể biết được những sinh viên đang chậm tiến độ học tập, những sinh viên đang bị mức cảnh cáo học tập để từ đó đưa ra giải pháp, tư vấn học tập phù hợp đối với các sinh viên.
Ngoài ra, hiện nay các vấn đề về đăng ký làm đồ án, đăng ký thực tập cho sinh viên đang được diễn ra một cách thủ công dưới hình thức điền form truyền thống.
Điều đó dẫn tới tốn rất nhiều thời gian để giảng viên sắp xếp, phản hồi, ... Vì vậy, em thực hiện đề tài "Phân tích, thiết kế hệ thống quản lý học tập và đăng ký đồ án, thực tập của sinh viên"
với mong muốn giảm bớt khối lượng công việc cho cả Viện cũng như của sinh viên. Ngoài ra, hệ thống cũng giúp sinh viên dễ dàng hơn trong quá trình
tìm hiểu cũng như chọn đề tài đồ án, vị trí thực tập của mình.

\section{Yêu cầu của bài toán}
%sửa
Hệ thống cho phép quản lý 10000 người dùng bao gồm sinh viên, giảng viên, cán bộ và người quản trị hệ thống.
Mỗi sinh viên sẽ sử dụng tài khoản microsoft mail trường cấp của mình để đăng ký tài khoản người dùng, sau đó có thể liên kết với
các tài khoản xã hội khác (google gmail, facebook). \\

Hệ thống quản lý sinh viên bao gồm các chức năng:
\begin{itemize}
	\item Đăng ký, đăng nhập, đăng xuất khỏi hệ thống
	\item Cập nhật thông tin cá nhân (điểm học tập)
	\item Tra cứu thông tin đề tài đồ án, giảng viên hướng dẫn
	\item Tra cứu thông tin công ty thực tập
	\item Đăng ký đồ án
	\item Tạo CV cá nhân dành cho sinh viên
	\item Đăng ký thực tập
	\item Thống kê kết quả học tập dành cho giảng viên
\end{itemize}

\section{Tác nhân của hệ thống}
\begin{tabular}{|c|c|l|}
	\hline
	\thead{STT} & \thead{Tác nhân} & \thead{Chức năng}                                                        \\
	\hline
	1           & Quản trị viên    & \makecell[l]{- Quản trị hệ thống.                                        \\ - Phân quyền người dùng. \\ - Cấp lại mật khẩu cho người dùng.}\\
	\hline
	2           & Cán bộ           & \makecell[l]{- Quản lý danh sách sinh viên.                              \\ - Quản lý form đăng ký trả về.} \\
	\hline
	3           & Giảng viên       & \makecell[l]{- Quản lý danh sách sinh viên theo lớp.}                         \\
	\hline
	4           & Sinh viên        & \makecell[l]{- Đăng nhập, đăng ký, đăng xuất, cập nhật thông tin.        \\ - Sử dụng hệ thống, thực hiện điền form. \\ - Tìm kiếm, tra cứu thông tin.}\\
	\hline
\end{tabular}

\section{Mô tả hệ thống}
\subsection{Dành cho quản trị viên}
\begin{itemize}
	\item Đăng nhập, đăng xuất khỏi hệ thống
	\item Phân quyền người dùng, quản trị hệ thống
\end{itemize}
\subsection{Dành cho sinh viên}
\begin{itemize}
	\item Mỗi sinh viên có thể sử dụng tài khoản Microsoft do nhà trường cung cấp (đuôi @sis.hust.edu.vn hoặc @hust.edu.vn) để đăng ký tài khoản sau đó liên kết với các tài khoản khác (Google, Facebook, ...). Sau đó có thể sử dụng một trong các tài khoản này để đăng nhập vào hệ thống. Khi đăng nhập thành công sẽ hiển thị giao diện trang chủ.
	\item Sinh viên có thể cập nhật thông tin cá nhân (tên, lớp, trạng thái học tập, điểm CPA, số tín chỉ đã qua, số tín chỉ nợ, ...) tại trang cá nhân.
	\item Sinh viên có thể tìm kiếm thông tin về đề tài đồ án (đồ án I, đồ án II, đồ án tốt nghiệp), thông tin giảng viên hướng dẫn.
	\item Sinh viên có thể thực hiện nhập thông tin đăng ký đồ án, xem trạng thái đăng ký, chờ kết quả trả về.
	\item Sinh viên có thể tìm kiếm các thông tin thực tập: vị trí, công ty, lĩnh vực.
	\item Sinh viên có thể thực hiện điền các thông tin cá nhân khác (giới thiệu bản thân, kinh nghiệm, các dự án tham gia, ...) để từ đó hệ thống sẽ lưu dữ liệu và xuất ra bản CV để từ đó có thể kết nối với doanh nghiệp cũng như nhận thực tập một cách dễ dàng hơn.
	\item Sinh viên đăng ký thực tập đính kèm bản CV cá nhân. Cuối mỗi kì thực tập, sinh viên sẽ cập nhật kết quả thực tập của mình tại doanh nghiệp.
\end{itemize}
\subsection{Dành cho giảng viên}
\begin{itemize}
	\item Đăng nhập/đăng xuất khỏi hệ thống.
	\item Khi đăng nhập thành công sẽ hiển thị trang dành cho giảng viên.
	\item Giảng viên có thể xem thống kê về học tập của các sinh viên lớp mình phụ trách. Để từ đó có thể liên hệ, tư vấn học tập cũng như đưa ra những giải pháp kịp thời cho sinh viên.
	\item Giảng viên có thể tạo mới các đề tài đồ án, thông tin thực tập doanh nghiệp của mình để cung cấp cho sinh viên. Các thông tin này sẽ được cán bộ duyệt trước khi trả về đến phía sinh viên.
\end{itemize}
\subsection{Dành cho cán bộ}
\begin{itemize}
	\item Tạo/duyệt các tài khoản giảng viên.
	\item Khi đăng nhập thành công sẽ hiển thị trang dành cho cán bộ.
	\item Quản lý các hoạt động của sinh viên: mở form đăng ký, duyệt form, sắp xếp giảng viên cho các sinh viên nhận đồ án, thực tập.
	\item Quản lý các hoạt động của giảng viên: duyệt các đề tài đồ án, thông tin thực tập doanh nghiệp do giảng viên tạo mới.
\end{itemize}