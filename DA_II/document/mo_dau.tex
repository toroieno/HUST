\chapter*{Mở đầu}
\addcontentsline{toc}{chapter}{Mở đầu}
Ngày nay cùng với sự phát triển mọi mặt của xã hội, ngành Công nghệ thông tin (CNTT) đã trở thành một phần không thể thiếu trong cuộc sống hàng ngày của chúng ta, thay đổi cách chúng ta sống, làm việc và tương tác với nhau.
Lĩnh vực này đang phát triển nhanh chóng và tác động đến nhiều khía cạnh khác nhau trong cuộc sống của chúng ta.

Một trong những lợi ích chính của CNTT là tăng hiệu quả và năng suất. Các hệ thống và công cụ CNTT có thể tự động hoá nhiều tác vụ thủ công, giúp tiết kiệm thời gian và tăng năng suất.
Điều này cũng có thể dẫn đến tiết kiệm chi phí cho các doanh nghiệp. Ví dụ, nhiều công ty đã triển khai phần mềm tự động hoá các tác vụ thông thường như nhập và xử lý dữ liệu, cho phép nhân viên tập trung vào các tác vụ chiến lược và phức tạp hơn.
Ngoài ra, CNTT cũng giúp việc truy cập và chia sẻ thông tin trở nên dễ dàng hơn. Điều này đã cho phép các cá nhân và tổ chức đưa ra quyết định sáng suốt hơn. 
Một ví dụ khác là các nhà cung cấp dịch vụ chăm sóc sức khoẻ có thể sử dụng hồ sơ y tế điện tử để truy cập thông tin bệnh nhân một cách nhanh chóng và dễ dàng, trong khi các doanh nghiệp có thể sử dụng các công cụ phân tích dữ liệu để phân tích lượng lớn dữ liệu và thu được thông tin chuyên sâu có thể giúp họ đưa ra quyết định tốt hơn.

Là một trường thuộc xếp hạng đầu về công nghệ của Việt Nam, Đại học Bách Khoa Hà Nội từ lâu đã áp dụng công nghệ trong việc giảng dạy cũng như phục vụ cho công tác quản lý sinh viên.
Bài toán quản lý sinh viên nhằm giải quyết và đáp ứng một cách hiệu quả các nhu cầu về mặt quản lý thông tin của các sinh viên trong trường.
Việc này sẽ làm giảm bớt sức lao động của con người, tiết kiệm được thời gian, độ chính xác cao, gọn nhẹ và tiện lợi hơn rất nhiều so với việc làm thủ công quản lý trên giấy tờ như trước đây.
Tuy nhiên, trong hệ thống của nhà trường vẫn còn một số mặt chưa thực sự được quản lý một cách hợp lý như vấn đề học tập của sinh viên, vấn đề đăng ký đồ án cũng như thực tập.

Vì vậy, em quyết định thực hiện đề tài \textbf{"Phân tích, thiết kế hệ thống web quản lý học tập và đăng ký đồ án, thực tập của sinh viên"}.
Là một đề tài mang tính thực tiễn cao, phần nào đưa ra được những nhận xét, đánh giá tổng thể từ đó đưa ra được hệ thống mới có thể khắc phục được những hạn chế mà hệ thống cũ còn tồn tại.

Mặc dù có nhiều cố gắng nhưng đồ án của em không thể tránh khỏi những thiếu sót kính mong các thầy cô đưa ra ý kiến để em có thể cải tiến hoàn thiện một cách tốt nhất.
Em rất mong được sự góp ý của thầy cô.

Em xin trân thành cảm ơn.

\newpage
Đề tài: \textbf{Phân tích, thiết kế hệ thống web quản lý học tập và đăng ký đồ án, thực tập của sinh viên}.

Đề tài được xây dựng trên các ngôn ngữ chính là:
\begin{itemize}
  \item Ngôn ngữ lập trình PHP và JavaScript.
  \item Hệ quản trị cơ sở dữ liệu PostgreSQL.
\end{itemize}
Bố cục của báo cáo gồm 3 chương:

\textbf{Chương 1: Tổng quan về đề tài}. Chương này em giới thiệu sơ lược về vấn đề quản lý sinh viên còn đang hạn chế của Viện ta hiện nay,
khảo sát thực trạng cũng như đánh giá hệ thống cũ. Từ đó đề xuất hệ thống mới tốt hơn.

\textbf{Chương 2: Phân tích hệ thống}. Sau khi đã thực hiện khảo sát hệ thống, em tiến hành phân tích nghiệp vụ với các \textit{Biểu đồ phân cấp chức năng}, \textit{Biểu đồ usecase}, \textit{Biểu đồ lớp}, \textit{Biểu đồ tuần tự}.

\textbf{Chương 3: Thiết kế hệ thống}. Cuối cùng sau khi đã thu thập đầy đủ thông tin về mặt khảo sát, em thực hiện \textit{Thiết kế cơ sở dữ liệu} và \textit{Thiết kế giao diện của website}.
