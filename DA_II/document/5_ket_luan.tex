\chapter*{Kết luận}
\addcontentsline{toc}{chapter}{Kết luận}
Mô hình "quản lý sinh viên" là một mô hình không mới nhưng nó vẫn còn rất nhiều khía cạnh hạn chế mà ta còn cần quan tâm cải tiến.
Quản lý sinh viên đề cập đến quá trình quản lý và tổ chức học tập và phát triển các nhân của sinh viên trong trường học.
Quản lý sinh viên hiệu quả là điều cần thiết để tạo ra một môi trường học tập tích cực, thúc đẩy thành công trong học tập và thúc đẩy sự phát triển của mỗi cá nhân sinh viên.
Việc hỗ trợ các cá nhân sinh viên, xác định những sinh viên có thể gặp khó khăn về mặt học tập hoặc cá nhân và cung cấp cho họ sự hỗ trợ và nguồn lực cần thiết để thành công.

Trong phạm vi đồ án này, em đã tìm hiểu và phân tích, thiết kế mô hình quản lý việc học tập của sinh viên cũng như việc đăng ký đồ án, thực tập. Cụ thể, đồ án đã thực hiện được một số nội dung như sau:
  \begin{itemize}
    \item Hiểu được quy trình nghiệp vụ trong hệ thống quản lý sinh viên.
    \item Phân tích và xác định chức năng của hệ thống.
    \item Phân tích và thiết kế cơ sở dữ liệu lưu trữ thông tin của hệ thống.
    \item Xây dựng được các chức năng quản trị thông tin học tập của sinh viên.
    \item Quản lý việc đăng ký đồ án, thực tập của sinh viên.
  \end{itemize}


\noindent\textbf{Những hạn chế của đồ án:}
\begin{itemize}
  \item Hạn chế về mặt hình thức: giao diện chưa hoàn thiện, thiết kế chưa chuyên nghiệp.
  \item Cơ sở dữ liệu chưa phong phú và đa dạng.
\end{itemize}

\noindent\textbf{Một số hướng phát triển tiếp theo:}
  \begin{itemize}
    \item Phát triển giao diện hoàn thiện, thân thiện với cả sinh viên và giảng viên.
    \item Tiếp tục nâng cấp, cập nhật thông tin cũng như các tính năng mà người dùng yêu cầu.
    \item Thử nghiệm và cải thiện tốc độ thực thi của ứng dụng.
  \end{itemize}