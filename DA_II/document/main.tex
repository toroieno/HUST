\documentclass[12pt, a4paper]{report}
\usepackage[utf8]{vietnam}
\usepackage{array, makecell}
\renewcommand\theadfont{\bfseries}


\begin{document}
Hệ thống quản lý sinh viên khoa/viện trường đại học Bách khoa Hà Nội
% Yêu cầu chức năng (Function requiment): mô tả các chức năng mà phần mềm hệ thống cung cấp
% Yêu cầu phi chức năng (Non-Function requiment): mô tả các ràng buộc đặt lên dịch vụ và quá trình phát triển hệ thống (về chất lượng, về môi trường, chuẩn sử dụng, qui trình phát triển)
\chapter{Khảo sát hệ thống}
Mỗi năm trường đại học Bách khoa Hà Nội tuyển sinh khoảng gần 8000 sinh viên (số liệu năm 2022 ), với nhu cầu phục vụ các sinh viên một cách đầy đủ nhất\\
Công tác quản lý sinh viên (kết quả học tập) của sinh viên đóng vai trò hết sức quan trọng đối với hoạt động của một khoa trong các trường đại học.
\section{Hệ thống cũ}

\section{Yêu cầu hệ thống mới}
Hệ thống có thể phục vụ lượng lớn khoảng 2000
% (150*16) 
sinh viên các khoa(viện) cùng lúc, có các chức năng như:
\begin{itemize}
    \item Thể hiện được mô hình tổ chức sinh viên theo khoá, theo lớp, các ngành đào tạo
    \item Quản lý điểm (CPA) cập nhật qua các học kì
    \item Hệ thống có thể xuất ra các báo cáo thống kê tình trạng học tập sinh viên dựa trên số điểm (CPA), số tín chỉ nợ, đưa ra mức cảnh cáo
    \item Hệ thống có chức năng đăng ký form đồ án, thực tập
    \item Tìm kiếm vị trí thực tập về các mảng cụ thể, có thông tin liên hệ với giảng viên, doanh nghiệp.
\end{itemize}

\chapter{Đặc tả yêu cầu bài toán}
\section{Mô tả yêu cầu}
    \subsection{Yêu cầu hệ thống quản lý sinh viên}
        Xây dựng phần mềm quản lý thông tin sinh viên:
        \begin{enumerate}
            \item Phầm mềm có yêu cầu đăng ký/đăng nhập hệ thống.
            \item Phầm mềm có thông tin lưu trữ của sinh viên: CPA, số tín chỉ đã qua, số tín chỉ nợ.
            \item Phần mềm có thể thống kê, đánh giá theo các mức cảnh báo sinh viên.
            \item Chức năng đăng ký hướng dẫn đồ án, thực tập.
            \item Chức năng tạo CV
            \item Chức năng tìm kiếm giảng viên, đề tài đồ án, vị trí thực tập.
        \end{enumerate}
    
    \subsection{Hệ thống quản lý sinh viên}
    \begin{itemize}
        \item Chức năng đăng ký thành viên:
        \begin{itemize}
            \item Mỗi sinh viên đăng ký tài khoản mới với tài khoản microsoft mail trường cấp.
            \item Đăng ký trực tiếp từ giao diện khởi động hệ thống.
            \item Tài khoản đó sau khi đăng ký thành công có thể đăng nhập vào hệ thống.
            \item Khi đăng ký thành công thì mặc định đó là tài khoản sinh viên.
        \end{itemize}
        \item Chức năng đăng nhập/đăng xuất hệ thống có phân quyền người dùng:
        \begin{itemize}
            \item Tài khoản đăng nhập hệ thống với đúng tài khoản và mật khẩu mà hệ thống cung cấp.
            \item Có thể đăng nhập bằng tài khoản google gmail hoặc tài khoản facebook trước đó đã được liên kết với tài khoản đăng ký.
            \item Tài khoản đăng nhập nếu không còn nhu cầu sử dụng hệ thống hoặc cần đăng nhập tài khoản khác có thể tiến hành đăng xuất.
        \end{itemize}
        \item Chức năng lưu trữ thông tin:
        \begin{itemize}
            \item Sinh viên có thể cập nhật, lưu trữ các thông tin học tập của cá nhân: CPA, số tín chỉ đã qua, số tín chỉ nợ, thuộc lớp nào.
            \item Cán bộ giảng viên phụ trách có thể xem danh sách lớp và xác nhận sinh viên.
        \end{itemize}
        \item Chức năng cập nhật thông tin:
        \begin{itemize}
            \item Hệ thống cho phép người dùng thay đổi thông tin cá nhân của mình.
            \item Admin có thể cấp lại mật khẩu cho người dùng.
        \end{itemize}
        \item Chức năng thống kê, đưa ra mức cảnh báo:
        \begin{itemize}
            \item Hệ thống có thể đưa ra biểu đồ thống kê số lượng các sinh viên đang ở các mức cảnh báo 1, 2, 3 và các sinh viên chậm chương trình.
            \item Hệ thống có thể xuất dữ liệu ra file excel, pdf.
        \end{itemize}
        \item Chức năng đăng ký hướng dẫn đồ án, thực tập:
        \begin{itemize}
            \item Hệ thống cung cấp form đăng ký hướng dẫn đồ án, form đăng ký thực tập.
            \item Giáo vụ có thể kiểm tra các form đã submit về.
            \item Trưởng bộ môn có thể truy cấp để xếp thứ tự ưu tiên các sinh viên.
        \end{itemize}
        \item Chức năng tạo CV:
        \begin{itemize}
            \item Hệ thống cung cấp form để sinh viên điền các thông tin cá nhân của mình: giới thiệu bản thân, CPA, định hướng nghề nghiệp, kỹ năng hiện có,...
            \item Hệ thống nhận form submit và trả về một file word bản CV mẫu, sinh viên có thể tải xuống để xem, chỉnh sửa.
        \end{itemize}
        \item Chức năng tìm kiếm giảng viên, đề tài đồ án, vị trí thực tập:
        \begin{itemize}
            \item Hệ thống cho phép tìm kiếm từ khoá theo tên giảng viên, từ đó đưa ra các đề tài mà giảng viên hướng dẫn
            \item Hệ thống cung cấp danh sách các vị trí thực tập.
        \end{itemize}
    \end{itemize}
    
\section{Các tác nhân của hệ thống}
\begin{tabular}{|c|c|l|}
    \hline
    \thead{STT} & \thead{Tác nhân} & \thead{Chức năng} \\
    \hline
    1 & Admin & \makecell[l]{- Quản trị hệ thống. \\ - Phân quyền người dùng. \\ - Cấp lại mật khẩu cho người dùng.}\\
    \hline
    2 & Giáo vụ & \makecell[l]{- Quản lý danh sách sinh viên. \\ - Quản lý form trả về.} \\
    \hline
    3 & Giảng viên & \makecell[l]{- Quản lý danh sách sinh viên lớp.} \\
    \hline
    4 & Sinh viên & \makecell[l]{- Đăng nhập, cập nhật thông tin. \\ - Sử dụng hệ thống, thực hiện điền form.}\\
    \hline
    5 & Người dùng & \makecell[l]{- Người dùng hệ thống với chức năng đăng ký tài khoản.}\\
    \hline
\end{tabular}
\chapter{Biểu đồ phân cấp chức năng}

\end{document}
